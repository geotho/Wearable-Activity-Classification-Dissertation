\chapter{Preparation}
  This chapter details the work done before the main implementation of the project was started. It
  details the devices chosen to implement this project and the reasons for choosing them. It then
  discusses the existing libaries and APIs available for those devices and for the required data
  processing. Finally I describe software engineering techniques used.

  \section{Requirements analysis}
  
  \section{Introduction to signal processing}
  
  
  \section{Hardware devices}
    The success of this project depends partly on correct selection and understanding of the devices
    used to collect data. The devices are both required to contain accelerometers accessible to
    developers.
    
    Android devices were chosen as Android Wear was the most mature platform for developing with
    wearable devices at the time. It runs on the widest variety of devices and provides developer 
    access to its sensors.
    
    \subsection{Smartphone}
      The smartphone chosen for development was the Google Nexus 5. Smartphone technology has
      advanced to the point that many Android smartphones are homogeneous with respect to this
      project --- they all continue sufficient processing power, internal memory and an 
      accerometer capable of recording data.
      
      The Nexus 5 contains a tri-axial accelerometer capable of recording measurements $\pm2\si{g}$
      on each axis.
      
      %TODO: include tech specs of Nexus 5?
    
    \subsection{Smartwatch}
      The smartwatch chosen for development was the Samsung Galaxy Gear Live.
      
      Wearable devices not running Android typically run either Tizen, an open-source but not 
      widely adopted operating system --- such as the Samsung Galaxy Gear 2 --- or a proprietary 
      operating system that does not allow access to the raw accelerometer data, for example the 
      Jawbone Up. 
      
      There is more differentiation in smartwatches as there is in smartphones, with them varying
      not just in screen size but also in screen format (round or rectangular), battery life,
      charging facilities and sensors. Table~\ref{tab:smartwatch-features} presents differences
      between plausible devices.
    
  \section{Libraries and APIs}
      This project makes use of existing libraries and APIs for the data collection, data handling
      and classification aspects of the project. Preliminary investigation into each of these areas
      was conducted.
    
    \subsection{Android Sensor API}
      The Android platform Sensor API is implemented using a publisher-subscriber model. Listeners
      to a particular sensor must be registered and must implement an \texttt{onSensorChanged()}
      method. The \texttt{onSensorChanged()} method is called whenever the sensor reports a new 
      value. A \texttt{SensorEvent} object is provided, containing a timestamp at which the data was
      reported together with the new data.
      
      The rate at which \texttt{onSensorChanged()} is called is user-`suggested'; though it can be 
      specified by the user, it can also be altered by the Android system. In practice, this means
      that the difference in timestamps not constant but is approximately equal to the specified 
      delay. A histogram of timestamp differences for a particular 10 minute recording is given in 
      figure~\ref{fig:timestamp-differences}.
      
      Android provides both acceleration and linear acceration sensors, related by 
      $$\textrm{acceleration} = \textrm{linear acceleration} + \textrm{gravity}$$
      They each provide a timestamp represented as a long and three float values representing the 
      acceleration of each axis in \si{\metre\per\square\second} at that timestamp.   
      Table~\ref{tab:data-row} gives a graphical representation of the data returned.
      
      \begin{table}
        \begin{tabu} to \linewidth {|X[2,c] | X[c] | X[c] | X[c] |}
          \hline
          Timestamp & X acceleration & Y acceleration & Z acceleration \\
          \si{ns} & \si{\metre\per\square\second} & \si{\metre\per\square\second} & \si{\metre\per\square\second} \\
          Long & Float & Float & Float \\
          2 bytes & 1 byte & 1 byte & 1 byte \\
          \hline
        \end{tabu}
        \caption{Data from the accelerometer sensor provided to the \texttt{onSensorChanged()} 
            method.}
        \label{tab:data-row}
      \end{table}
      
      Curiously, the timestamp returned as part of the data is documented only as ``The time in nanosecond at which the event happened'' \cite{androidsensoreventapi}. Futher exploration reveals that the timestamp is not defined against any particular zero-base, but rather the time since the device was powered on \cite{androidissuedocumentationbug, androidissuehardwarebug}. The implication of this for the project is that while the timestamp can be relied on for intervals between measurements, it cannot be used between different sets of recordings or across devices.
      
      %TODO: which are the x y z values of the device's accelerometer?
      
    \subsection{ES Sensor Manager}
    \subsection{Android Wear Data API}
    \subsection{Scikit Learn}
  \section{Choice of tools}
  \section{Software engineering techniques}
  \section{Summary}
