% Template for a Computer Science Tripos Part II project dissertation
\documentclass[12pt,a4paper,twoside,openright]{report}
\usepackage[pdfborder={0 0 0}]{hyperref}    % turns references into hyperlinks
\usepackage{geometry}  % adjusts page layout
\usepackage{graphicx}  % allows inclusion of PDF, PNG and JPG images
\usepackage{verbatim}
\usepackage{docmute}   % only needed to allow inclusion of proposal.tex
\usepackage[style=numeric-comp, backend=bibtex]{biblatex}
\addbibresource{refs.bib}

\raggedbottom                           % try to avoid widows and orphans
\sloppy
\clubpenalty1000%
\widowpenalty1000%

\renewcommand{\baselinestretch}{1.1}    % adjust line spacing to make
                                        % more readable

\begin{document}



%%%%%%%%%%%%%%%%%%%%%%%%%%%%%%%%%%%%%%%%%%%%%%%%%%%%%%%%%%%%%%%%%%%%%%%%
% Title


\pagestyle{empty}

\rightline{\LARGE \textbf{Martin Richards}}

\vspace*{60mm}
\begin{center}
\Huge
\textbf{How to write a dissertation in \LaTeX} \\[5mm]
Computer Science Tripos -- Part II \\[5mm]
St John's College \\[5mm]
\today  % today's date
\end{center}

%%%%%%%%%%%%%%%%%%%%%%%%%%%%%%%%%%%%%%%%%%%%%%%%%%%%%%%%%%%%%%%%%%%%%%%%%%%%%%
% Proforma, table of contents and list of figures

\pagestyle{plain}

\chapter*{Proforma}

{\large
\begin{tabular}{ll}
Name:               & \bf Martin Richards                       \\
College:            & \bf St John's College                     \\
Project Title:      & \bf How to write a dissertation in \LaTeX \\
Examination:        & \bf Computer Science Tripos -- Part II, July 2001  \\
Word Count:         & \bf 1587\footnotemark[1]
                      (well less than the 12000 limit)  \\
Project Originator: & Dr M.~Richards                    \\
Supervisor:         & Dr Markus Kuhn                    \\ 
\end{tabular}
}
\footnotetext[1]{This word count was computed
by \texttt{detex diss.tex | tr -cd '0-9A-Za-z $\tt\backslash$n' | wc -w}
}
\stepcounter{footnote}


\section*{Original Aims of the Project}

To write a demonstration dissertation\footnote{A normal footnote without the
complication of being in a table.} using \LaTeX\ to save
student's time when writing their own dissertations. The dissertation
should illustrate how to use the more common \LaTeX\ constructs. It
should include pictures and diagrams to show how these can be
incorporated into the dissertation.  It should contain the entire
\LaTeX\ source of the dissertation and the makefile.  It should
explain how to construct an MSDOS disk of the dissertation in
Postscript format that can be used by the book shop for printing, and,
finally, it should have the prescribed layout and format of a diploma
dissertation.


\section*{Work Completed}

All that has been completed appears in this dissertation.

\section*{Special Difficulties}

Learning how to incorporate encapulated postscript into a \LaTeX\
document on both Ubuntu Linux and OS X.
 
\newpage
\section*{Declaration}

I, [Name] of [College], being a candidate for Part II of the Computer
Science Tripos [or the Diploma in Computer Science], hereby declare
that this dissertation and the work described in it are my own work,
unaided except as may be specified below, and that the dissertation
does not contain material that has already been used to any substantial
extent for a comparable purpose.

\bigskip
\leftline{Signed [signature]}

\medskip
\leftline{Date [date]}

\tableofcontents

\listoffigures

\newpage
\section*{Acknowledgements}

This document owes much to an earlier version written by Simon Moore
\cite{poop}.  His help, encouragement and advice was greatly 
appreciated.

%%%%%%%%%%%%%%%%%%%%%%%%%%%%%%%%%%%%%%%%%%%%%%%%%%%%%%%%%%%%%%%%%%%%%%%
% now for the chapters

\pagestyle{headings}

\chapter{Introduction}
  This dissertation describes the implementation and evaluation of an activity classifier using 
  accelerometer data captured simutaneously from a smartphone and a smartwatch.  
  
  The classifier using data from both sources outperforms a classifier using only smartphone data,
  and the classifier that uses only smartphone data outperforms a classifier using only smartwatch 
  data.
  \section{Motivation}
  \label{sec:intro-motivation}
    Wearable devices are set to become the next big technology trend. Wrist-worn wearables, 
    including smartwatches, formed the majority of the 21m wearable devices sold year. Analysts
    predict the Apple Watch will sell between 20m and 40m in its first nine months 
    \cite{econapplewatch}.
    
    One of the primary appeals of wearables is their ability to sense. Like smartphones before them,
    smartwatches will enhance the ability to collect data about people. This data is important to
    consumers, who purchase specialised wearables to measure activity, sleep patterns and 
    caloric intake. The data's research potential is also laudable --- Apple's ResearchKit will
    allow medical researchers to access data about their patients with greater ease than ever
    before \cite{appleresearchkit}.
        
    Accurate activity classification therefore has many academic and commercial applications. To be
    marketable, activity classification solutions must use current consumer devices. Though 
    rudimentary activity classification is available on Android smartphones, an approach that
    utilises simutaneous collection from a smartphone and smartwatch has not been investigated in
    any detail.
     
    This dissertation details the implemenation of accelerometer data collection using current 
    consumer devices (an Android smartphone and Android Wear smartwatch), classifies a user's 
    activities and compares this classification accuracy to using only smartphone data and using 
    only smartwatch data. 
    
  \section{Challenges}
  \label{sec:intro-challenges}
    This project requires knowledge of a variety of disparate areas in computer science. 
    
    Writing software for mobile devices requires knowledge of their paradigms and nuances.
    Mobile devices are also subject to battery life and computational power constraints and   
    particular care must be taken to build a solution that works in practice.  
    A project that utilises built-in sensors also requires an understanding of the features and 
    limitations of those sensors and good knowledge in the APIs that are provided to access them.

    The sensors also output data at a high rate and care must be taken to correctly handle the
    performance and concurrency issues that may arise. Storage and transfer of large amounts of
    raw data, especially on a memory-limited device such as a smartwatch, also requires special
    consideration. 
    
    The data processing aspects of the project will require an understanding of digital signal 
    processing, Fourier methods, artificial intelligence and machine learning, and statistics.
  
  \section{Related Work}
  \label{sec:intro-relatedwork}
    Activity classification using accelerometer data from body-mounted devices is not a new topic: 
    Bao et al. \cite{bao2004activity} detect physicial activities using five biaxial acceleometers
    worn on different parts of the body. 
    
    Long et al. \cite{long2009single} use a single tri-axial accelerometer placed on the wrist and 
    use it to achieve an $80\%$ activity classification accuracy in five activities. One of the
    more interesting highlights of Long et al. is that only $50\%$ of all cycling in correctly
    classified.
    
    


\chapter{Preparation}

This chapter is empty!

\chapter{Implementation}

empty

\section{Tables}

\begin{samepage}
Here is a simple example\footnote{A footnote} of a table.

\begin{center}
\begin{tabular}{l|c|r}
Left      & Centred & Right \\
Justified &         & Justified \\[3mm]
%\hline\\%[-2mm]
First     & A       & XXX \\
Second    & AA      & XX  \\
Last      & AAA     & X   \\
\end{tabular}
\end{center}

\noindent
There is another example table in the proforma.
\end{samepage}

\chapter{Evaluation}

\section{Printing and binding}

Use a ``duplex'' laser printer that can print on both sides to print
two copies of your dissertation. Then bind them, for example using the
comb binder in the Computer Laboratory Library.

\chapter{Conclusion}

I hope that this rough guide to writing a dissertation is \LaTeX\ has
been helpful and saved you time.


%%%%%%%%%%%%%%%%%%%%%%%%%%%%%%%%%%%%%%%%%%%%%%%%%%%%%%%%%%%%%%%%%%%%%
% the bibliography
\addcontentsline{toc}{chapter}{Bibliography}
\printbibliography
%%%%%%%%%%%%%%%%%%%%%%%%%%%%%%%%%%%%%%%%%%%%%%%%%%%%%%%%%%%%%%%%%%%%%
% the appendices
\appendix

\chapter{Latex source}

\section{diss.tex}
{\scriptsize\verbatiminput{diss.tex}}

\section{proposal.tex}
{\scriptsize\verbatiminput{proposal.tex}}

\chapter{Makefile}

\section{makefile}\label{makefile}
{\scriptsize\verbatiminput{makefile.txt}}

\section{refs.bib}
{\scriptsize\verbatiminput{refs.bib}}


\chapter{Project Proposal}

% \input{proposal}

\end{document}
