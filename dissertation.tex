% Template for a Computer Science Tripos Part II project dissertation
\documentclass[12pt,a4paper,twoside,openright]{report}
\usepackage[pdfborder={0 0 0}]{hyperref}    % turns references into hyperlinks
\usepackage[margin=25mm]{geometry}  % adjusts page layout
\usepackage{graphicx}  % allows inclusion of PDF, PNG and JPG images
\usepackage{verbatim}
\usepackage{docmute}   % only needed to allow inclusion of proposal.tex
\usepackage[style=numeric-comp, backend=bibtex]{biblatex}
\addbibresource{refs.bib}

\raggedbottom                           % try to avoid widows and orphans
\sloppy
\clubpenalty1000%
\widowpenalty1000%

\renewcommand{\baselinestretch}{1.1}    % adjust line spacing to make
                                        % more readable

\begin{document}



%%%%%%%%%%%%%%%%%%%%%%%%%%%%%%%%%%%%%%%%%%%%%%%%%%%%%%%%%%%%%%%%%%%%%%%%
% Title


\pagestyle{empty}

\rightline{\LARGE \textbf{Martin Richards}}

\vspace*{60mm}
\begin{center}
\Huge
\textbf{How to write a dissertation in \LaTeX} \\[5mm]
Computer Science Tripos -- Part II \\[5mm]
St John's College \\[5mm]
\today  % today's date
\end{center}

%%%%%%%%%%%%%%%%%%%%%%%%%%%%%%%%%%%%%%%%%%%%%%%%%%%%%%%%%%%%%%%%%%%%%%%%%%%%%%
% Proforma, table of contents and list of figures

\pagestyle{plain}

\chapter*{Proforma}

{\large
\begin{tabular}{ll}
Name:               & \bf Martin Richards                       \\
College:            & \bf St John's College                     \\
Project Title:      & \bf How to write a dissertation in \LaTeX \\
Examination:        & \bf Computer Science Tripos -- Part II, July 2001  \\
Word Count:         & \bf 1587\footnotemark[1]
                      (well less than the 12000 limit)  \\
Project Originator: & Dr M.~Richards                    \\
Supervisor:         & Dr Markus Kuhn                    \\ 
\end{tabular}
}
\footnotetext[1]{This word count was computed
by \texttt{detex diss.tex | tr -cd '0-9A-Za-z $\tt\backslash$n' | wc -w}
}
\stepcounter{footnote}


\section*{Original Aims of the Project}

To write a demonstration dissertation\footnote{A normal footnote without the
complication of being in a table.} using \LaTeX\ to save
student's time when writing their own dissertations. The dissertation
should illustrate how to use the more common \LaTeX\ constructs. It
should include pictures and diagrams to show how these can be
incorporated into the dissertation.  It should contain the entire
\LaTeX\ source of the dissertation and the makefile.  It should
explain how to construct an MSDOS disk of the dissertation in
Postscript format that can be used by the book shop for printing, and,
finally, it should have the prescribed layout and format of a diploma
dissertation.


\section*{Work Completed}

All that has been completed appears in this dissertation.

\section*{Special Difficulties}

Learning how to incorporate encapulated postscript into a \LaTeX\
document on both Ubuntu Linux and OS X.
 
\newpage
\section*{Declaration}

I, [Name] of [College], being a candidate for Part II of the Computer
Science Tripos [or the Diploma in Computer Science], hereby declare
that this dissertation and the work described in it are my own work,
unaided except as may be specified below, and that the dissertation
does not contain material that has already been used to any substantial
extent for a comparable purpose.

\bigskip
\leftline{Signed [signature]}

\medskip
\leftline{Date [date]}

\tableofcontents

\listoffigures

\newpage
\section*{Acknowledgements}

This document owes much to an earlier version written by Simon Moore
\cite{poop}.  His help, encouragement and advice was greatly 
appreciated.

%%%%%%%%%%%%%%%%%%%%%%%%%%%%%%%%%%%%%%%%%%%%%%%%%%%%%%%%%%%%%%%%%%%%%%%
% now for the chapters

\pagestyle{headings}

\chapter{Introduction}

\section{Overview of the files}

This document consists of the following files:

\begin{itemize}
\item \texttt{makefile} --- The makefile for the dissertation and
                         Project Proposal
\item \texttt{diss.tex} --- The dissertation
\item \texttt{proposal.tex}  --- The project proposal 
\item \texttt{figs} -- A directory containing diagrams and pictures
\item \texttt{refs.bib} --- The bibliography database
\end{itemize}

\section{Building the document}

This document was produced using \LaTeXe which is based upon
\LaTeX\cite{Lamport86}.  To build the document you first need to
generate \texttt{diss.aux} which, amongst other things, contains the
references used.  This if done by executing the command:

\texttt{pdflatex diss}

\noindent
Then the bibliography can be generated from \texttt{refs.bib} using:

\texttt{bibtex diss}

\noindent
Finally, to ensure all the page numbering is correct run \texttt{pdflatex}
on \texttt{diss.tex} until the \texttt{.aux} files do not change.  This
usually takes 2 more runs.

\subsection{The makefile}

To simplify the calls to \texttt{pdflatex} and \texttt{bibtex}, 
a makefile has been provided, see Appendix~\ref{makefile}. 
It provides the following facilities:

\begin{description}

\item\texttt{make} \\
 Display help information.

\item\texttt{make proposal.pdf} \\
 Format the proposal document as a PDF.

\item\texttt{make view-proposal} \\
 Run \texttt{make proposal.pdf} and then display it with a Linux PDF viewer
 (preferably ``okular'', if that is not available fall back to ``evince'').

\item\texttt{make diss.pdf} \\
 Format the dissertation document as a PDF.

\item\texttt{make count} \\
Display an estimate of the word count.

\item\texttt{make all} \\
Construct \texttt{proposal.pdf} and \texttt{diss.pdf}.

\item\texttt{make pub} \\ Make \texttt{diss.pdf}
and place it in my \texttt{public\_html} directory.

\item\texttt{make clean} \\ Delete all intermediate files except the
source files and the resulting PDFs. All these deleted files can
be reconstructed by typing \texttt{make all}.

\end{description}


\section{Counting words}

An approximate word count of the body of the dissertation may be
obtained using:

\texttt{wc diss.tex}

\noindent
Alternatively, try something like:

\verb/detex diss.tex | tr -cd '0-9A-Z a-z\n' | wc -w/


\chapter{Preparation}

This chapter is empty!


\chapter{Implementation}

\section{Verbatim text}

Verbatim text can be included using \verb|\begin{verbatim}| and
\verb|\end{verbatim}|. I normally use a slightly smaller font and
often squeeze the lines a little closer together, as in:

{\renewcommand{\baselinestretch}{0.8}\small
\begin{verbatim}
GET "libhdr"
 
GLOBAL { count:200; all  }
 
LET try(ld, row, rd) BE TEST row=all
                        THEN count := count + 1
                        ELSE { LET poss = all & ~(ld | row | rd)
                               UNTIL poss=0 DO
                               { LET p = poss & -poss
                                 poss := poss - p
                                 try(ld+p << 1, row+p, rd+p >> 1)
                               }
                             }
LET start() = VALOF
{ all := 1
  FOR i = 1 TO 12 DO
  { count := 0
    try(0, 0, 0)
    writef("Number of solutions to %i2-queens is %i5*n", i, count)
    all := 2*all + 1
  }
  RESULTIS 0
}
\end{verbatim}
}

\section{Tables}

\begin{samepage}
Here is a simple example\footnote{A footnote} of a table.

\begin{center}
\begin{tabular}{l|c|r}
Left      & Centred & Right \\
Justified &         & Justified \\[3mm]
%\hline\\%[-2mm]
First     & A       & XXX \\
Second    & AA      & XX  \\
Last      & AAA     & X   \\
\end{tabular}
\end{center}

\noindent
There is another example table in the proforma.
\end{samepage}

\section{Simple diagrams}

Simple diagrams can be written directly in \LaTeX.  For example, see
figure~\ref{latexpic1} on page~\pageref{latexpic1} and see
figure~\ref{latexpic2} on page~\pageref{latexpic2}.

\begin{figure}
\setlength{\unitlength}{1mm}
\begin{center}
\begin{picture}(125,100)
\put(0,80){\framebox(50,10){AAA}}
\put(0,60){\framebox(50,10){BBB}}
\put(0,40){\framebox(50,10){CCC}}
\put(0,20){\framebox(50,10){DDD}}
\put(0,00){\framebox(50,10){EEE}}

\put(75,80){\framebox(50,10){XXX}}
\put(75,60){\framebox(50,10){YYY}}
\put(75,40){\framebox(50,10){ZZZ}}

\put(25,80){\vector(0,-1){10}}
\put(25,60){\vector(0,-1){10}}
\put(25,50){\vector(0,1){10}}
\put(25,40){\vector(0,-1){10}}
\put(25,20){\vector(0,-1){10}}

\put(100,80){\vector(0,-1){10}}
\put(100,70){\vector(0,1){10}}
\put(100,60){\vector(0,-1){10}}
\put(100,50){\vector(0,1){10}}

\put(50,65){\vector(1,0){25}}
\put(75,65){\vector(-1,0){25}}
\end{picture}
\end{center}
\caption{A picture composed of boxes and vectors.}
\label{latexpic1}
\end{figure}

\begin{figure}
\setlength{\unitlength}{1mm}
\begin{center}

\begin{picture}(100,70)
\put(47,65){\circle{10}}
\put(45,64){abc}

\put(37,45){\circle{10}}
\put(37,51){\line(1,1){7}}
\put(35,44){def}

\put(57,25){\circle{10}}
\put(57,31){\line(-1,3){9}}
\put(57,31){\line(-3,2){15}}
\put(55,24){ghi}

\put(32,0){\framebox(10,10){A}}
\put(52,0){\framebox(10,10){B}}
\put(37,12){\line(0,1){26}}
\put(37,12){\line(2,1){15}}
\put(57,12){\line(0,2){6}}
\end{picture}

\end{center}
\caption{A diagram composed of circles, lines and boxes.}
\label{latexpic2}
\end{figure}



\section{Adding more complicated graphics}

The use of \LaTeX\ format can be tedious and it is often better to use
encapsulated postscript (EPS) or PDF to represent complicated graphics.
Figure~\ref{epsfig} and~\ref{xfig} on page \pageref{xfig} are
examples. The second figure was drawn using \texttt{xfig} and exported in
{\tt.eps} format. This is my recommended way of drawing all diagrams.


\begin{figure}[tbh]
\centerline{\includegraphics{figs/cuarms.pdf}}
\caption{Example figure using encapsulated postscript}
\label{epsfig}
\end{figure}

\begin{figure}[tbh]
\vspace{4in}
\caption{Example figure where a picture can be pasted in}
\label{pastedfig}
\end{figure}


\begin{figure}[tbh]
\centerline{\includegraphics{figs/diagram.pdf}}
\caption{Example diagram drawn using \texttt{xfig}}
\label{xfig}
\end{figure}


\chapter{Evaluation}

\section{Printing and binding}

Use a ``duplex'' laser printer that can print on both sides to print
two copies of your dissertation. Then bind them, for example using the
comb binder in the Computer Laboratory Library.

\section{Further information}

See the Unix Tools notes at

\url{http://www.cl.cam.ac.uk/teaching/current-1/UnixTools/materials.html}


\chapter{Conclusion}

I hope that this rough guide to writing a dissertation is \LaTeX\ has
been helpful and saved you time.


%%%%%%%%%%%%%%%%%%%%%%%%%%%%%%%%%%%%%%%%%%%%%%%%%%%%%%%%%%%%%%%%%%%%%
% the bibliography
\addcontentsline{toc}{chapter}{Bibliography}
\printbibliography
%%%%%%%%%%%%%%%%%%%%%%%%%%%%%%%%%%%%%%%%%%%%%%%%%%%%%%%%%%%%%%%%%%%%%
% the appendices
\appendix

\chapter{Latex source}

\section{diss.tex}
{\scriptsize\verbatiminput{diss.tex}}

\section{proposal.tex}
{\scriptsize\verbatiminput{proposal.tex}}

\chapter{Makefile}

\section{makefile}\label{makefile}
{\scriptsize\verbatiminput{makefile.txt}}

\section{refs.bib}
{\scriptsize\verbatiminput{refs.bib}}


\chapter{Project Proposal}

\input{proposal}

\end{document}
