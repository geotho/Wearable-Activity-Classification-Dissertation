\chapter{Introduction}
  This dissertation describes the implementation and evaluation of an activity classifier using 
  accelerometer data captured simutaneously from a smartphone and a smartwatch.  
  
  The classifier using data from both sources outperforms a classifier using only smartphone data,
  and the classifier that uses only smartphone data outperforms a classifier using only smartwatch 
  data.
  \section{Motivation}
  \label{sec:intro-motivation}
    Wearable devices are set to become the next big technology trend. Wrist-worn wearables, 
    including smartwatches, formed the majority of the 21m wearable devices sold year. Analysts
    predict the Apple Watch will sell between 20m and 40m in its first nine months 
    \cite{econapplewatch}.
    
    One of the primary appeals of wearables is their ability to sense. Like smartphones before them,
    smartwatches will enhance the ability to collect data about people. This data is important to
    consumers, who purchase specialised wearables to measure activity, sleep patterns and 
    calorific intake. The data's research potential is also laudable --- Apple's ResearchKit will
    allow medical researchers to access data about their patients with greater ease than ever
    before \cite{appleresearchkit}.
        
    Accurate activity classification therefore has many academic and commercial applications. To be
    marketable, activity classification solutions must use current consumer devices. Though 
    rudimentary activity classification is available on Android smartphones, an approach that
    utilises simutaneous collection from a smartphone and smartwatch has not been investigated in
    any detail.
     
    For that reason, this dissertation details the implemenation of accelerometer data collection using current 
    consumer devices (an Android smartphone and Android Wear smartwatch), classifies a user's 
    activities and compares this classification accuracy to using only smartphone data and using 
    only smartwatch data. 
    
  \section{Challenges}
  \label{sec:intro-challenges}
    This project requires knowledge of a variety of disparate areas in computer science. 
    
    Writing software for mobile devices requires knowledge of their paradigms and nuances.
    Mobile devices are also subject to battery life and computational power constraints and   
    particular care must be taken to build a solution that works in practice.  
    A project that utilises built-in sensors also requires an understanding of the features and 
    limitations of those sensors and good knowledge in the APIs that are provided to access them.

    The sensors also output data at a high rate and care must be taken to correctly handle the
    performance and concurrency issues that may arise. Storage and transfer of large amounts of
    raw data, especially on a memory-limited device such as a smartwatch, also requires special
    consideration. 
    
    The data processing aspects of the project will require an understanding of digital signal 
    processing, Fourier methods, artificial intelligence and machine learning, and statistics.

  \section{Related Work}
  \label{sec:intro-relatedwork}
    Activity classification using accelerometer data from body-mounted devices is an active area of
    research. I highlight three papers and discuss their similarity to this problem. Summaries of
    their work are found in Table~\ref{tab:intro-relatedwork-comparison}.
    
    Unlike many previous investigations into this topic, this project differs by:
      \begin{enumerate}
        \item being implemented on consumer hardware; and
        \item using devices that are not fixed to the body (in the case of the phone).
      \end{enumerate}

    Bao \emph{et al.} \cite{bao2004activity} detect physicial activities using five biaxial accelerometers
    worn on different parts of the body: hip, wrist, ankle, arm and thigh. They find that accuracy  
    is not significantly reduced when using just thigh and wrist accelerometers. Futhermore,
    recognition rates for thigh and wrist data resulted in the highest recognition accuracy among
    all pairs of accelerometers, with over a $25\%$ improvement over the best single accelerometer
    results. This supports the viability of this project, with the improvement of being able to use 
    triaxial accelerometers found in consumer smartphones and smartwatches.
    
    Long \emph{et al.} \cite{long2009single} use a single triaxial accelerometer placed on the wrist and use it to achieve an $80\%$ activity classification accuracy in five activities. However, only $50\%$ of all cycling is correctly classified. Bao \emph{et al.} achieve an accuracy of $>92\%$ by using thigh and wrist data. This would suggest that wrist data alone is not sufficient to accurately classify certain types of activity. Cycling requires periodic leg motion (pedalling) while the hands and wrists move comparably little. Many of the features of motion used in activity classification require frequency domain analysis, and so data that contains periodic motion will be easier to recognise.
    
    Atallah \emph{et al.} \cite{atallah2010sensor} focus on two important facets of accelerometer-based activity classification: sensor location and useful features. Much like Bao \emph{et al.} they use seven sensors on the chest, arm, wrist, waist, knee, ankle and ear. Of their analysed features, the averaged entropy over three axes, the mean of the pairwise cross-covariance of axes and the energy of a 0.2 Hz window around the main frequency divided by total energy are all highlighted as being highly ranked for distinguishing activities. However, this study neglects to use a decision tree classifier in its classification, recommended by both Bao \emph{et al.} and Long \emph{et al.}
    \begin{table}[p]
      {\tabulinesep=1.2mm
      \begin{tabu} to \linewidth { p{2cm} X X X}
        \hline
          & \textbf{Bao \emph{et al.} \cite{bao2004activity}}
          & \textbf{Long \emph{et al.} \cite{long2009single}}
          & \textbf{Atallah \emph{et al.} \cite{atallah2010sensor}} \\
        \hline
          \textbf{Activities}
          & Walking, \newline sitting \& relaxing, \newline standing, \newline watching TV,
              \newline running, \newline stretching, \newline scrubbing, \newline folding laundry, 
              \newline brushing teeth, \newline riding elevator, \newline carrying items, 
              \newline computer work, \newline eating or drinking, \newline reading, 
              \newline bicycling, \newline strength-training, \newline vacuuming, 
              \newline lying down, \newline climbing stairs, \newline riding escalator
          & Walking, \newline running, \newline cycling, \newline driving, \newline sports
          & Lying down, \newline preparing food, \newline eating and drinking, 
              \newline socialising, \newline reading, \newline getting dressed, 
              \newline corridor walking, \newline treadmill walking, \newline vacuuming, 
              \newline wiping tables, \newline corridor running, \newline treadmill running, 
              \newline cycling, \newline sitting down and getting up, 
              \newline lying down and getting up \\
        \hline
          \textbf{Features}
          & Mean, energy, \newline correlation, \newline entropy
          & Standard deviation, \newline entropy, \newline orientation variation
          & Mean, variance, \newline root mean square, \newline entropy, \newline correlation, \newline range, energy, \newline primary frequency, \newline skewness, kurtosis \\
        \hline
          \textbf{Classifiers}
          & Decision table, \newline nearest neighbour, \newline decision tree, \newline naive Bayes
          & Decision tree, \newline principle component analysis, \newline naive Bayes 
          & K-nearest neighbors, \newline naive Bayes \\
        \hline
          \textbf{Overall accuracy}
          & $84\%$ & $80\%$ & N/A \\
        \hline
      \end{tabu}}
      \caption{Prior work on accelerometer-based activity classification}
      \label{tab:intro-relatedwork-comparison}
    \end{table}
      
    
    
