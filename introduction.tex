\chapter{Introduction}
  This dissertation describes the implementation and evaluation of an activity classifier using 
  accelerometer data captured simutaneously from a smartphone and a smartwatch. 
  
  The classifier using data from both sources outperforms a classifier using only smartphone data,
  and the classifier that uses only smartphone data outperforms a classifier using only smartwatch 
  data.
  \section{Motivation}
  \label{sec:intro-motivation}
    Wearable devices are set to become the next big technology trend. Wrist-worn wearables, 
    including smartwatches, formed the majority of the 21m wearable devices sold year. Analysts
    predict the Apple Watch will sell between 20m and 40m in its first nine months 
    \cite{econapplewatch}.
    
    One of the primary appeals of wearables is their ability to sense. Like smartphones before them,
    smartwatches will enhance the ability to collect data about people. This data is important to
    consumers, who purchase specialised wearables to measure activity, sleep patterns and 
    caloric intake. The data's research potential is also laudable --- Apple's ResearchKit will
    allow medical researchers to access data about their patients with greater ease than ever
    before \cite{appleresearchkit}.
        
    Accurate activity classification therefore has many academic and commercial applications. To be
    marketable, activity classification solutions must use current consumer devices. This 
    dissertation details the implemenation of accelerometer data collection using current consumer 
    devices (an Android smartphone and Android Wear smartwatch), classifies a user's activities and 
    compares this classification accuracy to using only smartphone data and using only smartwatch 
    data. 
    
  \section{Challenges}
  \label{sec:intro-challenges}
  
  \section{Related Work}
  \label{sec:intro-relatedwork}
